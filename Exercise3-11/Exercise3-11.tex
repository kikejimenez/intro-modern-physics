\documentclass[0pt, letterpaper]{article}
\usepackage[utf8]{inputenc}
\usepackage{fourier}
\usepackage{geometry}
\usepackage{graphicx}
\usepackage{amssymb}
\usepackage{amsmath}
\usepackage{amsthm}
\newcommand{\commentedbox}[2]{%
  \mbox{
    \begin{tabular}[t]{@{}c@{}}
    $\boxed{\displaystyle#1}$\\
    #2
    \end{tabular}%
  }%
}
 \geometry{
 a4paper,
 total = {150mm,237mm},
 left = 25mm,
 top = 25mm,
 tmargin = 10mm}

\title{Exercise 3.11 - Special Relativity. Woodhouse}
\author{García Parra Joel Alberto}
\date{Mayo 2019}
\usepackage[spanish]{babel}
\usepackage[utf8]{inputenc}
\begin{document}

\maketitle

\section{Exercise 3.11}
Show that for the constant acceleration wordline in this section, the proper acceleration is $cd\theta/d\tau$, where $\theta$ is the rapidity (Excercise 2.14).
\section{Solution}
\begin{itemize}
    \item First of all we take the equation that gives the Exercise 2.14: 
        \begin{equation}
            c\tanh{\theta}=v
        \end{equation}
    Note that this equation refers to a time $t$ in the frame.
    \par Now, let's derivate this equation with respect of the propper time $\tau$.
        \begin{equation}
            \frac{dv}{d\tau} = c\frac{d(\tanh{\theta})}{d\tau}
        \end{equation}
    \item In \textbf{(2)}, we can multiplie by $1=\frac{d\theta}{d\theta}$ and get:
        \begin{equation}
        \begin{split}
            \frac{d(\tanh{\theta})}{d\tau}(1)= & c\frac{d(\tanh{\theta})}{d\tau}\left(\frac{d\theta}{d\theta}\right)\\
            & = c\frac{d\tanh{\theta}}{d\theta}\left(\frac{d\theta}{d\tau}\right)
        \end{split}
        \end{equation}
    \item We can take the part of $\frac{d(\tanh{\theta})}{d\theta}$ and actually derivate it, but first we put $\tanh{\theta}$ in his exponential form to derivate it without using any trigonometric identity:
        \begin{equation}
            \tanh{\theta} = \frac{e^{\theta}-e^{-\theta}}{e^{\theta}+e^{-\theta}}
        \end{equation}
    \par Now, deriving with respect of $\theta$:
        \begin{equation}
        \begin{split}
            \frac{d}{d\theta}\left(\frac{e^{\theta}-e^{-\theta}}{e^{\theta}+e^{-\theta}}\right) = & \frac{\left(e^{\theta}+e^{-\theta}\right)\left(e^{\theta}+e^{-\theta}\right)-\left(e^{\theta}-e^{-\theta}\right)\left(e^{\theta}-e^{-\theta}\right)}{\left(e^{\theta}+e^{-\theta}\right)^2} \\
            & = \frac{\left(e^{\theta}+e^{-\theta}\right)^2-\left(e^{\theta}-e^{-\theta}\right)^2}{\left(e^{\theta}+e^{-\theta}\right)^2} \\
            & = 1 - \left(\frac{e^{\theta}-e^{-\theta}}{e^{\theta}+e^{-\theta}}\right)^2 \\
            & \text{Remmember the (4) equation} \\
            & = 1 - \left(\tanh{\theta}\right)^2
        \end{split}
        \end{equation}
    \item With this result, we can express $\frac{dv}{d\tau}$ in the next form:
        \begin{equation}
            \frac{dv}{d\tau} = c\frac{d\theta}{d\tau}\commentedbox{\left(1-tanh^2\theta\right)}{$dv/d\tau$}
        \end{equation}
    \item Using the definition of propper acceleration in this frame, we remmember:
        \begin{equation}
            a = \frac{1}{1-v^2/c^2}\frac{dv}{d\tau}
        \end{equation}
    \item But, we have defined $v$ at the beginning of the problem, so, let's put that       in the equation:
        \begin{equation*}
            v=c\tanh{\theta}
        \end{equation*}
        Where if we separate the term $v/c$ and square:
        \begin{equation}
            v^2/c^2 = \tanh^2{\theta}
        \end{equation}
    \item We have another look to the propper acceleration:
        \begin{equation*}
            a = \frac{1}{1-\commentedbox{\tanh^2{\theta}}{$v^2/c^2$}}\frac{dv}{d\tau}
        \end{equation*}
    \item Note that we already have $\frac{dv}{d\theta}$, so let's put it in:
        \begin{equation}
            a= \frac{1}{1-\tanh^2{\theta}}\commentedbox{\left(c\frac{d\theta}{d\tau}\left(1-\tanh^2{\theta}\right)\right)}{$dv/d\tau$}
        \end{equation}
    This equation easily become in to what we have been looking for:
        \begin{equation}
            \commentedbox{a = c\frac{d\theta}{d\tau}}{}
        \end{equation}
\end{itemize}
\end{document}
