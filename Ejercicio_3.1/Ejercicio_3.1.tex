%---------------------------------------------------------------------
%Inicio
\documentclass[12pt, A4,spanish,]{report}

%Paquetes
\usepackage{minted}
\usepackage{verbatim}
\usepackage{indentfirst}
\usepackage{ragged2e}
\usepackage{parskip}
\usepackage{graphicx}
\usepackage{geometry}
\usepackage[spanish]{babel}
\geometry {a4paper ,  top = 20 mm , right = 20mm, left = 2cm}
% Idioma español
\selectlanguage{spanish}
\usepackage[utf8]{inputenc}
% Idioma español
\usepackage[spanish]{babel}
\selectlanguage{spanish}
\usepackage[utf8]{inputenc}
%--------------------------------------------------------------
% Comienzo del documento
\begin{document}

%--------------------------------------------------------------
\section*{Problema 3.1 Woodhouse}
\large\justify
\textbf{Muestre que si $X$ y $Y$ son timelike o nulos, y $X^{0}>0$, $Y^{0}>0$, entonces $g(X,Y)>0$. (Pista: utilice la desigualdad de Cauchy-Schwarz.)}
\\
\\
Sabemos que:
\begin{itemize}
    \item Si $g(X,X)>0$
    \item Si $g(X,X)=0$ 
\end{itemize}
\\
\\
Entonces, de acuerdo al enunciado del problema:

    \[g(X,X)\geq0\]

    \[g(Y,Y)\geq0\]
Desarrollando $g(X,Y)$ tenemos:
\[g(X,Y)=X^{0}Y^{0}-X^{1}Y^{1}-X^{2}Y^{2}-X^{3}Y^{3}\]

Podemos reescribir esto como:
\begin{equation}
g(X,Y)=X^{0}Y^{0}-\vec{X}\cdot\vec{Y}     
\end{equation}
donde $\vec{X}$ y $\vec{Y}$ son vectores de 3 componentes.
\\
De la misma forma  para $g(X,X)$ y $g(Y,Y)$:

\begin{equation}
g(X,X)=(X^{0})^2-\abs{X}^2    
\end{equation}
\\

\begin{equation}
g(Y,Y)=(Y^{0})^2-\abs{\vec{Y}}^2    
\end{equation}
\\
Despejando $X^{0}$ y $Y^{0}$ de (2) y (3):
\begin{equation}
 X^{0}=\sqrt{g(X,X)+\abs{\vec{X}}^2}    
\end{equation}
\begin{equation}
 Y^{0}=\sqrt{g(Y,Y)+\abs{\vec{Y}}^2}    
\end{equation}
\\
Sustituyendo (4) y (5) en (1):
\begin{equation}
    g(X,Y)=\sqrt{g(X,X)+\abs{\vec{X}}^2}\sqrt{g(Y,Y)+\abs{\vec{Y}}^2}-\vec{X}\cdot\vec{Y}     
\end{equation}

\\
Partiendo de que $\abs{\vec{X}}\geq0$ y $\abs{\vec{Y}}\geq0$, podemos afirmar que: 

\[\sqrt{g(X,X)+\abs{\vec{X}}^2}\sqrt{g(Y,Y)+\abs{\vec{Y}}^2}\geq\abs{\vec{Y}}\abs{\vec{X}}\]

\\
Tomando en consideración esta desigualdad, reescribimos la ecuación (6) como:

\[g(X,Y)\geq\abs{\vec{Y}}\abs{\vec{X}}-\vec{X}\cdot\vec{Y}\]

\\
Aplicando la desigualdad de Cauchy-Schwartz, la cual establece que:

\[(\vec{X}\cdot\vec{Y})^2 \leq \abs{\vec{Y}}^2\abs{\vec{X}}^2 \]

Aplicando raíz cuadrada en ambos lados y desarrollando esta expresión:

\[\vec{X}\cdot\vec{Y} \leq \abs{\vec{Y}}\abs{\vec{X}} \]

\[  \abs{\vec{Y}}\abs{\vec{X}} - \vec{X}\cdot\vec{Y} \geq 0 \]

\\
Por lo tanto, podemos afirmar que $g(X,Y)\geq0$.
\\
\end{document}}



