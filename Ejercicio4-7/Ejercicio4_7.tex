\documentclass[0pt, letterpaper]{article}
\usepackage[utf8]{inputenc}
\usepackage{fourier}
\usepackage{geometry}
\usepackage{graphicx}
\usepackage{amssymb}
\usepackage{amsmath}
\usepackage{amsthm}

\newcommand{\commentedbox}[2]{%
  \mbox{
    \begin{tabular}[t]{@{}c@{}}
    $\boxed{\displaystyle#1}$\\
    #2
    \end{tabular}%
  }%
}

\newcommand{\vtor}[2]{
    \langle #1 , #2 \rangle
}

 \geometry{
 a4paper,
 total = {150mm,237mm},
 left = 25mm,
 top = 25mm,
 tmargin = 10mm}

\title{Ejercicio 4.7 - Special Relativity. Woodhouse}
\author{García Parra Pedro Antonio}
\date{Mayo 2019}
\usepackage[spanish]{babel}
\usepackage[utf8]{inputenc}
\begin{document}

\maketitle

\textbf{Ejercicio 4.7} En el ejemplo 4.2, muestra que si $\alpha$ y $\beta$ son, respectivamente, los angulos entre $\Vec{u}$ y $\Vec{v}$, y entre $\Vec{u}$ y $\Vec{w}$, entonces
\begin{equation}
    \cos^2\alpha = \frac{(\gamma(u) + 1)(\gamma(v) - 1)}{(\gamma(u) - 1)(\gamma(v) + 1)}
\end{equation}
Con una formula similar para $\cos^2\beta$. Dedusca que $\tan\alpha\tan\beta = 2/(1 + \gamma(u))$. ¿Cual es el resultado clasico correspondiente? (Hint: expresa $\gamma(w)^2\Vec{w}\cdot\Vec{w}$ en terminos de $\Vec{u}\cdot\Vec{v}$, $\gamma(u)$ y $\gamma(v)$.) \\

\textbf{Soluci\'on}\\

Para realizar este ejercicio se utilizar\'a la siguiente igualdad:
\begin{equation} \label{gammac2}
    \gamma(v)^2v^2 = c^2(\gamma(v)^2 - 1)
\end{equation}

Por conservaci\'on de 4-momento:
\begin{equation*}
    \begin{split}
        m\vtor{m}{c} + m\gamma(u)\vtor{c}{\Vec{u}} &= m\gamma(v)\vtor{v}{\Vec{v}} + m\gamma(w)\vtor{c}{\Vec{w}}\\
        \vtor{1 + \gamma(u)}{\gamma(u)\Vec{u}} &= \vtor{\gamma(v) + \gamma(w)}{\gamma(v)\Vec{v} + \gamma(w)\Vec{w}}
    \end{split}
\end{equation*}
De aqu\'i, podemos obtener, igualando componentes y despejando:

\begin{equation} \label{w}
    \gamma(w) = \gamma(u) - \gamma(v) + 1
\end{equation}
\begin{equation} \label{ww}
    \gamma(w)\Vec{w} = \gamma(u)\Vec{u} - \gamma(v)\Vec{v}
\end{equation}
Tomando el modulo cuadrado de \eqref{ww}:

\begin{equation*}
    \begin{split}
        \gamma(w)^2w^2 &= \gamma(u)^2u^2 + \gamma(v)^2v^2 - 2\gamma(u)\gamma(v)\Vec{u}\cdot\Vec{v}\\
        2\gamma(u)\gamma(v)\Vec{u}\cdot\Vec{v} &= \gamma(u)^2u^2 + \gamma(v)^2v^2 - \gamma(w)^2w^2
    \end{split}
\end{equation*}

Utilizando \eqref{gammac2} sobre la ecuaci\'on anterior:
\begin{equation} \label{2punto}
    2\gamma(u)\gamma(v)\Vec{u}\cdot\Vec{v} = c^2(\gamma(u)^2 + \gamma(v)^2 - \gamma(w)^2 - 1)
\end{equation}

Por otro lado, y utilizando \eqref{w}
\begin{equation*}
    \begin{split}
        \gamma(u)^2 + \gamma(v)^2 - \gamma(w)^2 - 1 &= \gamma(u)^2 + \gamma(v)^2 - (\gamma(u) - \gamma(v) + 1)^2 - 1 \\
        &= \gamma(u)^2 + \gamma(v)^2 - \gamma(u)^2 - \gamma(v)^2 - 1 + 2\gamma(u)\gamma(v) - 2\gamma(u) + 2\gamma(v) - 1 \\
        &= 2\gamma(u)\gamma(v) + 2\gamma(v) - 2\gamma(u) - 2 \\
        &= 2(\gamma(u)\gamma(v) + \gamma(v) - \gamma(u) - 1) \\
        &= 2(\gamma(u) + 1)(\gamma(v) - 1)
    \end{split}
\end{equation*}
Sustituyendo esto en \eqref{2punto} y desarrollando:
    \begin{equation*}
        \begin{split}
              2\gamma(u)\gamma(v)\Vec{u}\cdot\Vec{v} &= c^2[2(\gamma(u) + 1)(\gamma(v) - 1)] \\
              \Vec{u}\cdot\Vec{v} &= \frac{c^2(\gamma(u) + 1)(\gamma(v) - 1)}{\gamma(u)\gamma(v)} \\
              \cos\alpha &= \frac{c^2(\gamma(u) + 1)(\gamma(v) - 1)}{|u||v|\gamma(u)\gamma(v)} \\
              \cos^2\alpha &= \frac{c^4(\gamma(u) + 1)^2(\gamma(v) - 1)^2}{u^2v^2\gamma(u)^2\gamma(v)^2} 
        \end{split}
    \end{equation*}
Utilizando \eqref{gammac2} nuevamente:
    \begin{equation*}
        \begin{split}
            \cos^2\alpha &= \frac{c^4(\gamma(u) + 1)^2(\gamma(v) - 1)^2}{c^2(\gamma(u)^2 - 1)c^2(\gamma(v)^2 - 1)} \\ 
            \cos^2\alpha &= \frac{(\gamma(u) + 1)^2(\gamma(v) - 1)^2}{(\gamma(u) - 1)(\gamma(u) + 1)(\gamma(v) - 1)(\gamma(v) + 1)}
        \end{split}
    \end{equation*}
    
    \begin{equation*}
        \boxed{\cos^2\alpha = \frac{(\gamma(u) + 1)(\gamma(v) - 1)}{(\gamma(u) - 1)(\gamma(v) + 1)}}
    \end{equation*}
    
    Realizando el mismo procedimiento ahora en t\'erminos de $\Vec{u}$ y $\Vec{w}$ obtenemos que:
    \begin{equation}
        \cos^2\beta = \frac{(\gamma(u) + 1)(\gamma(w) - 1)}{(\gamma(u) - 1)(\gamma(w) + 1)}
    \end{equation}
    \\
    Para encontrar el valor de $\tan\alpha$ y $\tan\beta$ primero se deber\'a usar la relacion trigonom\'etrica de $\sin^2\theta = 1 - \cos^2\theta$ para obtener $\sin^2\alpha$ y $\sin^2\beta$. Posteriormente usamos que $\tan^2\theta = \frac{\sin^2\theta}{\cos^2\theta}$. As\'i obtenemos que:
    \begin{equation}\label{alpha}
        \tan^2\alpha = \frac{(\gamma(u) - 1)(\gamma(v) + 1) - (\gamma(u) + 1)(\gamma(v) - 1)}{(\gamma(u) + 1)(\gamma(v) - 1)}
    \end{equation}
        \begin{equation}\label{beta}
        \tan^2\beta = \frac{(\gamma(u) - 1)(\gamma(w) + 1) - (\gamma(u) + 1)(\gamma(w) - 1)}{(\gamma(u) + 1)(\gamma(w) - 1)}
    \end{equation}
    
    Multiplicando las ecuaciones \eqref{alpha} y \eqref{beta} y simplificando un poco, tenemos que:
    
    \begin{equation*}
        \begin{split}
            \tan^2\alpha\tan^2\beta &= \frac{4(\gamma(u) - \gamma(v)) \ (\gamma(u) - \gamma(w))}{(\gamma(u) - 1)^2(\gamma(v) - 1)(\gamma(u) - 1)}
        \end{split}
    \end{equation*}
    Utilizando la ecuaci\'on \eqref{w} podemos obtener:
    \begin{equation*}
        \tan^2\alpha\tan^2\beta = \frac{4}{(\gamma(u) + 1)^2}
    \end{equation*}
    Finalmente:
    \begin{equation}
        \boxed{\tan\alpha\tan\beta = \frac{2}{\gamma(u) + 1}}
    \end{equation}
\end{document}
